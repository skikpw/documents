\documentclass[a4paper,12pt]{article}
\usepackage[utf8]{inputenc}
\usepackage[polish]{babel}
\let\lll\undefined % do babelu, avoiding conflict
\usepackage{polski}
\selectlanguage{polish}
\usepackage[T1]{fontenc}
\frenchspacing
\usepackage{indentfirst}


\usepackage{amssymb}
\usepackage{amsmath}
\usepackage{amsfonts}
 
\usepackage[pdftex,dvipsnames]{xcolor} %red, green, blue, yellow, cyan, magenta, black, white
%polskie tabelki
\usepackage[nottoc]{tocbibind}

% Kropki w liczebnikach porzadkowych
\usepackage{titlesec}
\titlelabel{\thetitle.\quad}


\setlength{\parskip}{0.3\baselineskip}%
% \setlength{\parindent}{0pt}%

%polskie przypisy
\usepackage{caption}
\captionsetup{labelsep=period}
\captionsetup{font=small,labelfont=bf,labelsep=period,justification=centering}
\addto\captionspolish{\renewcommand{\figurename}{Rys.}}
\addto\captionspolish{\renewcommand{\tablename}{Tab.}}
\addto\captionspolish{\renewcommand{\seename}{zob.}}

\usepackage{geometry}
 \geometry{
 a4paper,
 %total={170mm,247mm}, % 170 257 20 20
 left=20mm,
 right=20mm,
 top=25mm,
 bottom=25mm,
 }   
   


\usepackage{listings}
\definecolor{mygreen}{RGB}{28,172,0} % color values Red, Green, Blue
\definecolor{mylilas}{RGB}{170,55,241}
% \newcounter{listing}
% \newenvironment{listing}[1][]{
% \refstepcounter{listing}\par\medskip \textbf{Kod~\thelisting. #1} \rmfamily}{\medskip}

\lstset{language=Matlab,%
    %basicstyle=\color{red},
    breaklines=true,%
    morekeywords={matlab2tikz},
    keywordstyle=\color{blue},%
    morekeywords=[2]{1}, keywordstyle=[2]{\color{black}},
    identifierstyle=\color{black},%
    stringstyle=\color{mylilas},
    commentstyle=\color{mygreen},%
    showstringspaces=false,%without this there will be a symbol in the places where there is a space
    numbers=left,%
    numberstyle={\tiny \color{black}},% size of the numbers
    numbersep=9pt, % this defines how far the numbers are from the text
    emph=[1]{for,end,break},emphstyle=[1]\color{red}, %some words to emphasise
    %emph=[2]{word1,word2}, emphstyle=[2]{style},    
    basicstyle=\small\ttfamily,
    frame=single,
    float
}


\usepackage{svg}
%SVG for the first time using inkscape: for i in *.svg; do inkscape -D -z --file=$i  --export-pdf="${i%.*}.pdf" --export-latex .; done




\usepackage{xargs}                      % Use more than one optional parameter in a new commands
% 
\usepackage[colorinlistoftodos,prependcaption,textsize=tiny]{todonotes}
\newcommandx{\unsure}[2][1=]{\todo[linecolor=red,backgroundcolor=red!25,bordercolor=red,#1]{#2}}
\newcommandx{\change}[2][1=]{\todo[linecolor=blue,backgroundcolor=blue!25,bordercolor=blue,#1]{#2}}
% \newcommandx{\info}[2][1=]{\todo[linecolor=OliveGreen,backgroundcolor=OliveGreen!25,bordercolor=OliveGreen,#1]{#2}}
% \newcommandx{\improvement}[2][1=]{\todo[linecolor=Plum,backgroundcolor=Plum!25,bordercolor=Plum,#1]{#2}}
% \newcommandx{\thiswillnotshow}[2][1=]{\todo[disable,#1]{#2}}

% \missingfigure[figwidth=6cm]{Testing a long text string}


\usepackage[section]{placeins} %\FloatBarrier
% \usepackage{hyperref}
   
\usepackage{fancyhdr}
\setlength{\headheight}{26pt}
\pagestyle{fancy}

\lhead{Sprawozdanie z grantu rektorskiego koła SKIK 2016} 	\rhead{\today\quad\includesvg[width=1.5cm]{logo_skik}}
\lfoot{}			\cfoot{}		\rfoot{strona \thepage/-}

\title{Projekt przedmiotu Sieci Neuronowe i Neurokomputery wydziału EiTI Politechniki Warszawskiej}
\author{Artur Dobrogowski}



\begin{document}


\begin{titlepage}
	\centering
	\includesvg[width=0.3\textwidth]{logo_pw}\par\vspace{1cm}
	{\scshape\Large Politechnika Warszawska\\Wydział Elektroniki i Technik Informacyjnych\\Instytut Radioelektroniki i Technik Multimedialnych \par}
	\vspace{1cm}
	{\huge\scshape\bfseries Sprawozdanie\\\par}
	{\scshape\Large z pracy naukowo-badawczej własnej\\ finansowanej z grantu rektorskiego za rok 2016\\\par}
	{\Large Umowa numer XXX Y/????/????\\\par}
	\vspace{1.5cm}
	{\Large\bfseries System pomiarowy na pokładzie kapsuły balonu stratosferycznego\\\par}
	\vspace{2cm}
	{\Large\itshape Studenckie koło inżynierii kosmicznej\par}
	\vfill
	Kierownik pracy:\par
	dr inż. Krzysztof \textsc{Kurek}

	\vfill

% Bottom of the page
	{\large \today\par}
\end{titlepage}

\section{Wprowadzenie}
% ** o balonach
Balon stratosferyczny pozwala na wyniesienie podczepionego pod nim ładunku na wysokość około 35 km nad poziom morza, co umożliwia realizację różnych pomiarów związanych z dolnymi warstwami atmosfery. Jednocześnie panujące na wysokości kilkudziesięciu km nad ziemią  niskie ciśnienie i niska temperatura pozwalają na test pracy wynoszonych układów w tak niekorzystnych warunkach. Realizacja misji balonowej może być więc prostym i tanim sposobem na sprawdzenie działania opracowanych układów w warunkach zbliżonych do środowiska przestrzeni kosmicznej.
Stratosferyczne misje balonowe są coraz częściej organizowane w Polsce, zarówno przez studentów np. ze Studenckiego Koła Astronautycznego PW, jak  również przez różne stowarzyszenia zrzeszające entuzjastów balonów i radioamatorów. 
Również SKIK, w swojej dotychczasowej działalności zrealizował kilka misji balonowych.
\subsection{Cel projektu}
Celem projektu jest zaprojektowanie i skonstruowanie platformy do misji balonowych i zbierania danych z górnych partii atmosfery. Opracowanie systemu pomiarowego w misji balonowej stanowi pewne odwzorowanie problemów z jakimi trzeba się zmierzyć podczas projektowania satelity, ponieważ w trakcie misji nie ma możliwości na poprawki ani usuwanie awarii. Jednym z najważniejszych aspektów jest komunikacja radiowa. Kluczowe jest śledzenie parametrów lotu i warunków panujących na zewnątrz układu, aby wiedzieć czy sprzęt prawidłowo funkcjonuje. W przypadku satelitów, po ukończeniu swojego zadania urządzenia ulegają zniszczeniu przy powrocie do atmosfery, natomiast w misji balonowej ważne jest odnalezienie urządzenia, odzyskanie danych zapisanych na dysku i zapewnienie bezpiecznego powrotu na ziemie.

Zadaniem systemu pokładowego kapsuły jest akwizycja danych w profilu wysokości oraz ich transmisja. Dane uzyskane z misji balonowych są nie do uzyskania w inny sposób, ponieważ samoloty nie wznoszą się tak wysoko, a satelity są w stanie dokonywać tylko pomiarów pośrednich.

\subsection{Czujniki}
Projektowany system zostanie wyposażymy w zestaw czujników, mierzących poziom dwutlenku węgla i wiglotności. Pomiar tych gazów jest istotny ze względu na modelowanie pogody jak i klimatu. 

Umożliwi to porównanie wyników, uzyskanych podczas planowanej misji balonowej z danymi z lat poprzednich i oszacowanie jak ludzka działalność przyczyniła się do zmian składu atmosfery. Poza tym zbieranie i publikowanie takich danych przez niezależne instytucje da lepszy wgląd w modelowanie klimatu i pomoże przyspieszyć akcje zapobiegające zmianom klimatycznym.

Platforma balonowa będzie się składać z:
\begin{itemize}
 \item komputera pokładowego Raspberry PI zero
 \item kamery nagrywającej lot do celów promowania koła
 \item moduł z czujnikiem do pomiaru stężenia dwutlenku węgla
 \item czujnik wilgotności
 \item czujnik ciśnienia
 \item czujnikami temperatury wewnątrz i na zewnątrz kapsuły
 \item modułu komunikacyjnego 433 MHz
 \item moduł zasilania
 \item modułu GPS
\end{itemize}


Oprócz tego w platformie będą umieszczone niezależne układy lokalizacyjne:
\begin{itemize}
 \item układ APRS będący tematem pracy magisterskiej mgr. Mateusza Walczyka
 \item lokalizator GPS-GSM z kartą SIM
\end{itemize}

 
%opis ogólny co mialo byc zrobione i po co
% misje balonowe, aprametry mierzone, co jest istotne, w sunkcji wysokosci
\section{Opracowanie projektu systemu czujników}
\missingfigure[figwidth=12cm]{Schemat blokowy/ideowy systemu}
\missingfigure[figwidth=12cm]{Schemat połączeń też ideowy}
\section{Realizacja systemu}
\section{Testowanie i integracja systemu}
\todo[inline]{Najpierw testowaliśmy osobno wszystkie moduły/czujniki niezależnie, np przy pomocy Arduino, a potem w systemie z Raspberry Pi.}
\section{Realizacja misji balonowej}

\end{document}

%  Wprowadzenie - ogólny opis co miało być zrobione i w jakim celu.
% Po co misje balonowe. Jakie parametry są mierzone. Jakie są istotne. Które dobrze byłoby mierzyć w funkcji wysokości - stąd propozycja misji balonowej.
% 2. Opracowanie projektu systemu pomiaru zanieczyszczeń 
% Struktura blokowa planowanego systemu - opis poszczególnych bloków, jakie czujniki były rozważane - ich krótki opis, co zostało finalnie wybrane, jakie są interfejsy do czujników, rozważania o zapewnieniu odpowiedniego środowiska do pomiaru na różnych wysokościach (uniezależnienie się od zmian ciśnienia i ewentualnie temperatury.
% Opis poszczególnych elementów ze schematu blokowego z rozdziału 2 - schematy ideowe, schematy płytek drukowanych, oprogramowanie pozwalające na podłączenie czujników do Rasberry PI, zasilanie, gondola, spadochron, element odblaskowy zapewniający widoczność na radarze (jeśli nie ma z przykładowych sprawozdaniach, które podesłałem, to powinien być opisany w pracy Mateusza Walczyka, do tego również  można powołać się na jego układ APRS i go opisać, podając odwołanie do jego pracy).
% 4. Integracja i testy systemu zanieczyszczeń
% Tu trzeba opisać zintegrowany system, czyli Rasberry PI z podłączonymi czujnikami, umieszczony w kapsule (tej która stoi w DS401), zdjęcia systemu, peleryna spadochronu (peleryna fryzjerska raczej zamarznie w -50 stopniach i nie spełni swojej funkcji), wyniki jakichś przykładowych pomiarów z podłączonych czujników,
% wyniki testów modułów radiowych, czyli przesłanie danych z Rasberry PI przez zakupione moduły radiowe.
% Tu można napisać, że ze względu na opóźnienia w realizacji prac projektowych, taki lot musiałby się odbyć na przełomie grudnia i stycznia, przy niskich temperaturach, więc został przełożony na wiosnę.