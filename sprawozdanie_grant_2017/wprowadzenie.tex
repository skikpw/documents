\section{Wprowadzenie}


Balon stratosferyczny pozwala na wyniesienia podczepionego pod nim ładunku na wysokość około 35 km nad poziom morza, na której panuje niskie ciśnienie i niska temperatura, czyli warunki zbliżone do środowiska przestrzeni kosmicznej. Stratosferyczne misje balonowe mogą być więc tanim sposobem na testowanie poprawnego działania układów zbudowanych przez studentów, a przewidzianych do pracy w przestrzeni kosmicznej. Jednocześnie umieszczenie różnych czujników w gondoli przyczepionej do balonu pozwala na pomiary różnych parametrów w atmosferze, w funkcji wysokości nad poziomem terenu. Misje takie są coraz częściej organizowane w Polsce, zarówno przez studentów np. ze Studenckiego Koła Astronautycznego PW, jak  również przez różne stowarzyszenia zrzeszające entuzjastów balonów i radioamatorów. Również SKIK, w swojej dotychczasowej działalności zrealizował kilka misji balonowych.

Kluczowym elementem misji balonowej jest przesył informacji na ziemię łączem radiowym, do czego jest potrzebna mobilna naziemna stacja odbiorcza, aby mieć możliwość odbioru danych przesyłanych niezależnie od miejsca startu misji. Ze względu na jakość odbioru, ważne jest aby stacja była z dala od źródeł zakłóceń i w pobliżu miejsca startu. W związku z tym istnieje potrzeba konstrukcji takiej stacji. Stacja mogłaby być wykorzystywana do odbioru transmisji, takich jak komunikatów z balonów meteorologicznych jak również w przyszłych misjach balonowych Studenckiego Koła Inżynierii Kosmicznej.
