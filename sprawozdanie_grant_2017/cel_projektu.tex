\section{Cel projektu}

Celem projektu jest opracowanie mobilnej stacji do łączności radiowej z balonem ze szczególnym zwróceniem uwagi na pasma radioamatorskie. Stacja składa się z:
\begin{itemize}
    \item statywu z elektronicznie sterowanymi antenami przy pomocy silników krokowych umożliwiających zmianę orientacji anten w płaszczyźnie azymutu i elewacji,
    \item wzmacniacza sygnału przychodzącego z anteny,
    \item zestawu filtrów pasmowo przepustowych,
    \item radia programowalnego (SDR – Software Defined Radio) do przetworzenia sygnału (wykorzystane zostało radio będące na stanie Laboratorium Technologii Kosmicznych WEiTI), do którego stworzone zostało oprogramowanie umożliwiające odbiór sygnałów
\end{itemize}  

	Opracowana stacja pozwala na odbiór sygnałów w paśmie radioamatorskim: UHF 430 MHz (opcjonalnie w paśmie VHF 140 MHz). 
	Stworzone oprogramowanie do radia programowalnego SDR umożliwia odbiór pakietów APRS – automatycznego systemu powiadamiania o pozycji, który jest zwykle wykorzystywany do lokalizacji balonów w misjach. Wykorzystanie radia programowalnego SDR pozwoli w przyszłości na proste dostosowanie stacji do odbioru innych typów sygnałów (które mogą być przesyłane z balonu) tylko przez zmiany oprogramowania, bez konieczności zmian sprzętowych (jeśli transmisja będzie realizowana w obsługiwanym paśmie częstotliwości).

	Wynikiem prac jest między innymi oprogramowanie pozwalające na odbiór danych systemu APRS i danych podczas trwania misji, z wykorzystaniem radia programowalnego SDR.
	Gotowa stacja stanowi atrakcyjny obiekt zainteresowania do zaprezentowania np. w trakcie targów studenckich kół naukowych KONIK, co pomoże w popularyzacji tematyki związanej z systemami satelitarnymi.
	Wiedza i doświadczenie inżynierskie  zdobyte przez członków Koła w trakcie realizacji projektu pozwolą myśleć o wzięciu udziału w edukacyjnych projektach balonowych organizowanych pod patronatem Europejskiej Agencji Kosmicznej.
